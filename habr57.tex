\secrel{Урок \cite{habr57}}\label{habr57tut}

\subsubsection{Старт}

После установки, и старта мы увидим экран:

\figx{fig/habr57/01.jpg}{height=0.95\textheight}

Здесь мы видим следующие объекты:

\begin{itemize}[nosep]
	\item сцена\ --- здесь и происходит всё действо, на данный момент она 
		пуста\note{есть только пол}
	\item слева видим блок с библиотекой моделей\ --- сверху папки, и под 
		ней отображается содержимое выбранной папки\note{выбраны 
		robots/non-mobile\ --- то есть стационарные роботы—манипуляторы}
\item далее отображается иерархия мира
\end{itemize}

Иерархия включает в себя корневой объект (мир), в котором находятся все объекты.

В нашем примере это:

\figx{fig/habr57/02.jpg}{height=0.35\textheight}

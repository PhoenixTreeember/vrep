\secrel{Урок habr57 \cite{habr57}}\label{habr57tut}

\subsubsection{Старт}

После установки, и старта мы увидим экран:
\figx{fig/habr57/01.jpg}{height=0.35\textheight}

Здесь мы видим следующие объекты:

\begin{itemize}[nosep]
	\item сцена\ --- здесь и происходит всё действо, на данный момент она 
		пуста\note{есть только пол}
	\item слева видим блок с библиотекой моделей\ --- сверху папки, и под 
		ней отображается содержимое выбранной папки\note{выбраны 
		robots/non-mobile\ --- то есть стационарные роботы—манипуляторы}
\item далее отображается иерархия мира
\end{itemize}

Иерархия включает в себя корневой объект (мир), в котором находятся все объекты.

В нашем примере это:

\figx{fig/habr57/02.jpg}{height=0.35\textheight}

Видим источники света, видим объект для реализации пола\note{твердая 
поверхность с текстурой}, и группу для камер.

Есть главный объект-скрипт, контролирующий сцену и всех объектов на ней, 
и у каждого объекта может быть свой скрипт\ --- внутренние скрипты реализованы 
на языке Lua.

Вверху и слева мы видим toolbar\ --- меню. Самой главной кнопкой является
кнопка \keys{\rms}\ (\textbf{Start Simulation})\ --- после которой стартует
симуляция сцены:

\figx{fig/habr57/03.jpg}{width=0.5\textwidth}

Сценарий работы следующий:
\begin{itemize}[nosep]
\item мы перетаскиваем с помощью DragAndDrop объекты из библиотеки моделей. 
\item корректируем их местоположение
\item настраиваем скрипты
\item стартуем симулятор
\item останавливаем симулятор
\end{itemize}

Попробуем что-нибудь на практике.

\subsubsection{Быстрый старт}

Попробуем оживить робота.

Для этого выбираем слева папку \file{robots/mobile}\ и в списке выбираем 
\file{Ansi}, захватываем, переносим на сцену и отпускаем, робот появляется 
на нашей сцене и появляется информация об авторе:

\figx{fig/habr57/04.jpg}{width=0.6\textwidth}

Теперь нажимаем на \keys{\rms}\ Start Simulation, и видим движение робота, и 
можем вручную управлять положением головы, рук\note{реализовано через панель
Custom User Interface}, \href{https://vimeo.com/122517749}{видео}.

Далее останавливаем симуляцию:

\figx{fig/habr57/05.jpg}{width=0.3\textwidth}

\subsubsection{Скрипт управления}

Можем открыть и увидеть код, который\note{управляет автономным передвижением 
робота}\ научил робота идти. Для этого на иерархии объектов, напротив модели 
\file{Asti}, дважды кликаем на иконке \menu{File}:

\figx{fig/habr57/06.jpg}{width=0.5\textwidth}

Lua программа, которая осуществляет движение робота:

\lstx{asti.lua}{lst/asti.lua}

\subsubsection{Другие модели}

Вы можете удалить модель\ --- для этого надо её выбрать на сцене или в дереве
сцены, и нажать на \keys{Del}. И можете попробовать посмотреть другие модели
в работе, у некоторых есть скрипты для автономной работы.

\subsubsection{Мобильные роботы}

\figx{fig/habr57/07.jpg}{width=0.8\textwidth}

\subsubsection{Стационарные роботы-манипуляторы}

\figx{fig/habr57/08.jpg}{width=0.9\textwidth}

\subsubsection{Примеры сцен}

Есть большое количество примеров сцен, которые поставляются сразу с программой.
Для этого надо выбрать в меню \menu{File>Open scene...}\ и там перейти в 
папку \file{C:/Program Files/V-REP3/V-REP\_PRO\_EDU/scenes}.


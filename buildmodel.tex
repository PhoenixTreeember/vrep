\secrel{Building a clean model}\label{buildmodel}

\cp{\url{http://www.coppeliarobotics.com/helpFiles/en/buildingAModelTutorial.htm}}

This tutorial will guide you step-by-step into building a clean simulation 
model, of a robot, or any other item. This is a very important topic, maybe 
the most important aspect, in order to have a nice looking, fast displaying, 
fast simulating and stable simulation model.

To illustrate the model building process, we will be building following 
manipulator:

\fig{\ Model of robotic manipulator}{fig/model/01.jpg}{height=0.5\textheight}

\subsubsection{Building the visible shapes}

When building a new model, first, we handle only the visual aspect of it: the 
dynamic aspect\note{its undelying even more simplified/optimized model}, joints, 
sensors, etc. will be handled at a later stage.

We could now directly create primitive shapes in \vrep\ with 
\menu{Menu>Add>Primitive shape>\ldots}. When doing this, we have the option to
create pure shapes, or regular shapes. Pure shape will be optimized for 
dynamic interaction, and also directly be dynamically enabled\note{i.e. fall,
collide, but this can be disabled at a later stage}. Primitive shapes will be 
simple meshes, which might not contain enough details or geometric accuracy 
for our application. Our other option in that case would be to import a mesh 
from an external application.

When importing CAD data from an external application, the most important is to 
make sure that the CAD model is not too heavy, i.e. doesn't contain too many 
triangles. This requirement is important since a heavy model will be slow in 
display, and also slow down various calculation modules that might be used at 
a later stage\note{e.g. minimum distance calculation, or dynamics}. Following 
example is typically a no-go\note{even if, as we will see later, there are means 
to simplify the data within \vrep}:

\fig{\\Complex CAD data (in solid and wireframe)}{fig/model/02.jpg}{height=0.5\textheight}

Above CAD data is very heavy: it contains many triangles\note{more than 
47'000}, which would be ok if we would just use a single instance of it in an 
empty scene. But most of the time you will want to simulate several instances 
of a same robot, attach various types of grippers, and maybe have those robots
interact with other robots, devices, or the environment. In that case, a 
simulation scene can quickly become too slow. Generally, we recommend to model 
a robot with no more than a total of 20'000 triangles, but most of the time 
5'000..10'000 triangles would just do fine as well. Remember: less is better,
in almost every aspect.

What makes above model so heavy? First, models that contain holes and small 
details will require much more triangular faces for a correct representation. 
So, if possible, try to remove all the holes, screws, the inside of objects,
etc. from your original model data. If you have the original model data
represented as parametric surfaces/objects, then it is most of the time a 
simple matter of selecting the items and deleting them\note{e.g. in 
SolidWorks}. The second important step is to export the original data with a 
limited precision: most CAD applications let you specify the level of details 
of exported meshes. It might also be important to export the objects in 
several steps, when the drawing consists of large and small objects; this is
to avoid having large objects too precisely defined\note{too many triangles}\
and small objects too roughly defined\note{too few triangles}: simply export 
large objects first\note{by adjusting the desired precision settings}, then 
small objects\note{by adjusting up precision settings}.

\vrep\ supports currently following CAD data formats: \file{OBJ}, \file{STL}, 
\file{DXF}, \file{3DS}\note{\win\ only}, and \file{Collada}. \file{URDF}\ is 
also supported, but not mentionned here since it is not a pure mesh-based
file format.

Now suppose that we have applied all possible simplifications as described in 
previous section. We might still end-up with a too heavy mesh after import:

\fig{Imported CAD data}{fig/model/03.jpg}{height=0.33\textheight}

You can notice that the whole robot was imported as a single mesh. We will 
see later how to divide it appropriately. Notice also the wrong orientation of
the imported mesh: best is to keep the orientation as it is, until the whole
model was built, since, if at a later stage we want to import other items that
are related to that same robot, they will automatically have the correct
position/orientation relative to the original mesh.

At this stage, we have several functions at our disposal, to simplify the mesh:
\begin{description}
	\item{\textbf{Automatic mesh division}}
	allows to generate a new shape for all 
	elements that are not linked together via a common edge. This does not 
	always work for the selected mesh, but is always worth a try, since 
	working on mesh elements gives us more control than if we had to work on 
	all elements at the same time. The function can be accessed with
	\menu{Menu>Edit>Grouping/Merging>Divide selected shapes}. Sometimes, a 
	mesh will be divided more than expected. In that case, simply merge 
	elements that logically belong together\note{i.e. that will have the same
	visual attributes and that are part of the same link}\ back into one 
	single shape \menu{Menu>Edit>Grouping/Merging>Merge selected shapes}.

	\item{\textbf{Extract the convex hull}} 
	allows to simplify the mesh by transforming
	it into a convex hull. The function can be accessed with
	\menu{Menu>Edit>Morph selection into convex shapes}.

	\item{\textbf{Decimate the mesh}}
	allows to reduce the number of triangles contained
	in the mesh. The function can be accessed with
	\menu{Menu>Edit>Decimate selected shape...}.

	\item{\textbf{Remove the inside of the mesh}}
	allows to simplify the mesh by 
	removing its inside. This function is based on vision sensors and might 
	give more or less satisfying results depending on the selected settings. 
	The function can be accessed with
	\menu{Menu>Edit>Extract inside of selected shape}.
\end{description}

There is no predefined order in which above functions can/should be 
applied\note{except for the first item in the list, which should always be
tried first}, it heavily depends on the geometry of the mesh we are trying to 
simplify. Following image illustrates above functions applied to the imported 
mesh\note{let's suppose the first item in the list didn't work for us}:

\fig{Convex hull, decimated mesh, and extracted inside}{fig/model/04.jpg}{height=0.95\textheight}\pagebreak

Notice how the convex hull doesn't help us at this stage. We decide to use the 
mesh decimation function first, and run the function twice in order to divide 
the number of triangles by a total of 50. Once that is done, we extract the 
inside of the simplified shape and discard it. We end-up with a mesh containing
a total of 2'660 triangles\note{the original imported mesh contained more than 
136'000 triangles\,!}. The number of triangles/vertices a shape contains can 
be seen in the shape geometry dialog. 2'660 triangles are extremely few 
triangles for a whole robot model, and the visual appearance might suffer a
little bit from it.

At this stage we can start to divide the robot into separate 
links\note{remember, we currently have only a single shape for the whole 
robot}. You can do this in two different ways:

\begin{description}
	\item{\textbf{Automatic mesh division}}
	this function, which was already described in previous section, will 
	inspect the shape and generate a new shape for all elements that are not 
	linked together via a common edge. This does not always work, but is 
	always worth a try. The function can be accessed with
	\menu{Menu>Edit>Grouping/merging>Divide selected shapes}.
	\item{\textbf{Manual mesh division}}
	via the the triangle edit mode, you can manually select the triangles 
	than logically belong together, then click \keys{Extract shape}. This will
	generate a new shape in the scene. Delete the selected triangles after 
	that operation.
\end{description}

In the case of our mesh, method 1 worked fine:

\fig{\\Divided mesh}{fig/model/05.jpg}{height=0.65\textheight}

Now, we could further refine/simplify individual shapes. Sometimes also, a 
shape might look better if its convex hull is used instead. Othertimes, you 
will have to use several of above's described techniques iteratively, in 
order to obtain the desired result. Take for instance following mesh:

\fig{Imported mesh}{fig/model/06.jpg}{height=0.4\textheight}

The problem with above's shape is that we cannot simplify it nicely, because 
of the holes it contains. So we have to go the more complicated way via the 
shape edit mode, where we can extract individual elements that logically 
belong to the same convex sub-entity. This process can take several iterations: 
we first extract 3 approximate convex elements. For now, we ignore the 
triangles that are part of the two holes. While editing a shape in the shape 
edit mode, it can be convenient to switch the visibility layers, in order to 
see what is covered by other scene items.

\fig{Step 1}{fig/model/07.jpg}{height=0.4\textheight}

We end up with a toal of three shapes, but two of them will need further 
improvement. Now we can erase the triangles that are part of the holes. 
Finally, we extract the convex hull individually for the 3 shapes, then merge 
them back together with \menu{Menu>Edit>Grouping/Merging>merge selected shapes}:

\fig{Step 2}{fig/model/08.jpg}{height=0.6\textheight}

In \vrep, we can enable/disable edge display for each shape. We can also 
specify an angle that will be taken into account for the edge display. A
similar parameter is the shading angle, that dictates how facetted the shape 
will display. Those parameters, and a few others such as the shape color, can
be adjusted in the shape properties. Remember that shapes come in various 
flavours. In this tutorial we have only dealt with simple shapes up to now: 
a simple shape has a single set of visual attributes\note{i.e. one color, one 
shading angle, etc.}. If you merge two shapes, then the result will be a
simple shape. You can also group shapes, in which case, each shape will 
retain its visual attributes.

In next step, we can merge elements that logically belong together\note{if they
are part of the same rigid element, and if they have the same visual attributes}.
Then we change the visual attributes of the various elements. The easiest ist 
to adjust a few shapes that have different colors and visual attributes, and 
if we name the color with a specific string, we can later easily 
programmatically change that color, also if the shape is part of a compound 
shape. Then, we select all the shapes that have the same visual attributes, 
then control-select the shape that was already adjusted, then click Apply to
selection, once for the Colors, once for the other properties, in the shape
properties: this transfers all visual attributes to the selected 
shapes\note{including the color name if you provided one}. We end up with 17 
individual shapes:

\fig{Adjusted visual attributes}{fig/model/09.jpg}{height=0.5\textheight}

Now we can group the shapes that are part of the same link with
\menu{Menu>Edit>Grouping/merging>Group selected shapes}. We end up with 7 
shapes: the base of the robot\note{or base of the robot's hierarchy tree}, 
and 6 mobile links. It is also important to correctly name your objects: you 
we do this with a double-click on the object name in the scene hierarchy. The 
base should always be the robot or model name, and the other objects should 
always contain the base object name, like: \verb|robot|\ (base), 
\verb|robot_link1|, \verb|robot_proximitySensor|, etc. By defaut, shapes will 
be assigned to visibility layer 1, but can be changed in the object common 
properties. By default, only visibility layers 1-8 are activated for the 
scene. We now have following (the model \verb|ResizableFloor_5_25|\ was
temporarily made invisible in the model properties dialog):

\fig{\\Individual elements compositn the robot}{fig/model/10.jpg}{height=0.8\textheight}

When a shape is created or modified, V-REP will automatically set its reference
frame position and orientation. A shape's reference frame will always be 
positioned at the shape's geometric center. The frame orientation will be 
selected so that the shape's bounding box remains as small as possible. This 
does not always look nice, but we can always reorient a shape's reference 
frame at any time. We now reorient the reference frames of all our created 
shapes with \menu{Menu>Edit>Reorient bounding box>with reference frame of 
world}. You have more options to reorient a reference frame in the shape 
geometry dialog.

\subsubsection{Building the joints}

Now we will take care of the joints/motors. Most of the time, we know the 
exact position and orientation of each of the joints. In that case, we simply 
add the joints with \menu{Menu>Add>Joints>\ldots}, then we can change their 
position and orientation with the coordinate and transformation dialogs. In 
other situations, we only have the Denavit-Hartenberg\note{i.e. D-H}\ parameters. 
In that case, we can build our joints via the tool model located in 
\file{Models/tools/Denavit-Hartenberg joint creator.ttm}, in the model browser.
Othertimes, we have no information about the joint locations and orientations. 
Then, we need to extract them from the imported mesh. Let's suppose this is 
our case. Instead of working on the modified, more approximate mesh, we open a 
new scene, and import the original CAD data again. Most of the time, we can 
extract meshes or primitive shapes from the original mesh. The first step is 
to subdivide the original mesh. If that does not work, we do it via the 
triangle edit mode. Let's suppose that we could divide the original mesh. We 
now have smaller objects that we can inspect. We are looking for revolute 
shapes, that could be used as reference to create joints at their locations, 
with the same orientation. First, remove all objects that are not needed. It 
is sometimes also useful to work across several opened scenes, for easier 
visualization/manipulation. In our case, we focus first on the base of the
robot: it contains a cylinder that has the correct position for the first 
joint. In the triangle edit mode, we have:

\fig{\\Robot base: normal and triangle edit mode visualization}{fig/model/11.jpg}{height=0.8\textheight}

We change the camera view via the page selector toolbar button, in order to 
look at the object from the side. The fit-to-view toolbar button can come in 
handy to correctly frame the object in edition. Then we switch to the vertex
edit mode and select all vertices that belong to the upper disc. Remember that
by switching some layers on/off, we can hide other objects in the scene. Then
we switch back to the triangle edit mode:

\fig{\\Selected upper disc, vertex edit mode (1 \& 2), triangle edit mode (3)}{fig/model/12.jpg}{height=0.8\textheight}

Now we click \menu{Extract cylinder}\note{\menu{Extract shape}\ would also work 
in that case}, this just created a cylinder shape in the scene, based on the 
selected triangles. We leave the edit mode and discard the changes. Now we add
a revolute joint with \menu{Menu>Add>Joint>Revolute}, keep it selected, then 
\keys{Ctrl-\lms}\ select the extracted cylinder shape. In the coordinate and 
transformation dialogs we click the Position/Translation button, then, in the 
Object/item position section, we click Apply to selection (the one under the 
edit boxes): this basically copies the x/y/z position of the cylinder to the 
joint. Both positions are now identical. We click the Orientation/Rotation 
button, then in the Object/item orientation section, we click Apply to 
selection: the orientation of our selected objects is now also the same. 
Sometimes, we will need to additionally rotate the joint about 90/180 degrees
around its own reference frame in order to obtain the correct orientation or
rotation direction. We could do that in the Object/item rotation opertations 
section if needed (in that case, do not forget to click the Own frame button).
In a similar way we could also shift the joint along its axis, or even do more
complex operations. This is what we have:

\fig{Joint in correct location, with the correct orientation}{fig/model/13.jpg}{height=0.5\textheight}

Now we copy the joint back into our original scene, and save it\note{do not 
forget to save your work on a regular basis\,! The undo/redo function is 
useful, but doesn't protect you against other mishaps}. We repeat above 
procedure for all the joints in our robot, then rename them. We also make all 
joints a little bit longer in the joint properties, in order to see them all. 
By defaut, joints will be assigned to visibility layer 2, but can be changed
in the object common properties. We assign now all joints to visibility layer
10, then temporarily enable visibility layer 10 for the scene to also 
visualize those joints (by default, only visibility layers 1-8 are activated 
for the scene). This is what we have\note{the model \file{ResizableFloor\_5\_25}\
was temporarily made invisible in the model properties dialog}:

\fig{\\Joints in correct configuration}{fig/model/14.jpg}{height=.5\textheight}

At this point, we could start to build the model hierarchy and finish the 
model definition. But if we want opur robot to be dynamically enabled, then 
there is an additional intermediate step:

\subsubsection{Building the dynamic shapes}

If we want our robot to be dynamically enabled, i.e. react to collisions, fall,
etc., then we need to create/configure the shapes appropriately: a shape can be:
\begin{description}
	\item{\textbf{dynamic or static}}
	a dynamic (or non-static) shape will fall and be influences by external
	forces/torques. A static (or non-dynamic) shape on the other hand, will 
	stay in place, or follow the movement of its parent in the scene hierarchy.
	\item{\textbf{respondable or non-respondable}}
	a respondable shape will cause a collision reaction with other respondable 
	shapes. They (and/or) their collider, will be influenced in their movement 
	if they are dynamic. On the other hand, non-respondable shapes will not 
	compute a collision response if they collide with other shapes.
\end{description}

Above two points are illustrated here. Respondable shapes should be as simple 
as possible, in order to allow for a fast and stable simulation. A physics 
engine will be able to simulate following 5 types of shapes with various 
degrees of speed and stability:

\begin{description}
\item{\textbf{Pure shapes}}
a pure shape will be stable and handled very efficiently by the physics engine. 
The draw-back is that pure shapes are limited in geometry: mostly cuboids, 
cylinders and spheres. If possible, use those for items that are in contact
with other items for a longer time\note{e.g. the feet of a humanoid robot, the 
base of a serial manipulator, the fingers of a gripper, etc.}. Pure shapes 
can be created with \menu{Menu>Add>Primitive shape}.
\item{\textbf{Pure compound shapes}}
a pure compound shape is a grouping of several pure shapes. It performs almost 
as well as pure shapes and shares similar properties. Pure compound shapes can
be generated by grouping several pure shapes
\menu{Menu>Edit>Grouping/Merging>Group selected shapes}.
\item{\textbf{Convex shapes}}
a convex shape will be a little bit less stable and take a little bit more 
computation time when handled by the physics engine. It allows for a more 
general geometry\note{only requirement: it need to be convex}\ than pure shapes. 
If possible, use convex shapes for items that are sporadically in contact 
with other items\note{e.g. the various links of a robot}. Convex shapes can be 
generated with \menu{Menu>Add>Convex hull of selection}\ or with
\menu{Menu>Edit>Morph selection into convex shapes}.
\item{\textbf{Compound convex shapes, or convex decomposed shapes}}
a convex decomposed shape is a grouping of several convex shapes. It performs 
almost as well as convex shapes and shares similar properties. Convex 
decomposed shapes can be generated by grouping several convex shapes
\menu{Menu>Edit>Grouping/Merging>Group selected shapes}, with
\menu{Menu>Add>Convex decomposition of selection\ldots}, or with
\menu{Menu>Edit>Morph selection into its convex decomposition\ldots}.
\item{\textbf{Random shapes}}
a random shape is a shape that is not convex nor pure. It generally has poor 
performance\note{calculation speed and stability}. Avoid using random shapes 
as much as possible.
\end{description}

So the order of preference would be: pure shapes, pure compound shapes, 
convex shapes, compound convex shapes, and finally random shapes. Make sure 
to also read this page. In case of the robot we want to build, we will make 
the base of the robot as a pure cylinder, and the other links as convex or 
convex decomposed shapes.

We could use the dynamically enabled shapes also as the visible parts of the 
robot, but that would probably not look good enough. So instead, we will build 
for each visible shape we have created in the first part of the tutorial a 
dynamically enabled counterpart, which we will keep hidden: the hidden part 
will represent the dynamic model and be exclusively used by the physics engine,
while the visible part will be used for visualization, but also for minimum 
distance calculations, proximity sensor detections, etc.

We select object robot, copy-and-paste it into a new scene\note{in order to 
keep the original model intact}\ and start the triangle edit mode. If object 
robot was a compound shape, we would first have had to ungroup it
\menu{Menu>Edit>Grouping/Merging>Ungroup selected shapes}\ then merge the 
individual shapes \menu{Menu>Edit>Grouping/Merging>Merge selected shapes}\
before being able to start the triangle edit mode. Now we select the few 
triangles that represent the power cable, and erase them. Then we select all 
triangles in that shape, and click \menu{Extract cylinder}. We can now leave
the edit mode and we have our base object represented as a pure cylinder:

\fig{\\Pure cylinder generation procedure, in the triangle edit mode}{fig/model/15.jpg}{height=.5\textheight}

We rename the new shape\note{with a double-click on its name in the scene 
hierarchy}\ as \verb|robot_dyn|, assign it to visibility layer 9, then copy 
it to the original scene. The rest of the links will be modelled as convex 
shapes, or compound convex shapes. We now select the first mobile 
link\note{i.e. object \file{robot\_link1}}\ and generate a convex shape from 
it with \menu{Menu>Add>Convex hull of selection}. We rename it to 
\verb|robot_link_dyn1|\ and assign it to visibility layer 9. When extracting
the convex hull doesn't retain enough details of the original shape, then you
could still manually extract several convex hulls from its composing elements,
then group all the convex hulls with
\menu{Menu>Edit>Grouping/Merging>Group selected shapes}. If that appears to
be problematic or time consuming, then you can automatically extract a convex
decomposed shape with \menu{Menu>Add>Convex decomposition of selection\ldots}:

\fig{\\Original shape, and convex shape pendant}{fig/model/16.jpg}{height=.4\textheight}

\fig{\\Original shape, and convex decomposed shape pendant}{fig/model/17.jpg}{height=.4\textheight}

We now repeat the same procedure for all remaining robot links. Once that is 
done, we attach each visible shape to its corresponding invisible dynamic
pendant. We do this by selecting first the visible shape, then via
\keys{Ctrl+\lms}\ selecting its dynamic pendant then
\menu{Menu>Edit>Make last selected object parent}. The same result can be 
achieved by dragging the visible shape onto its dynamic pendant in the scene
hierarchy:

\fig{\\Visible shapes attached to their dynamic pendants}{fig/model/18.jpg}{height=.5\textheight}

We still need to take care of a few things: first, since we want the dynamic 
shapes only visible to the physics engine, but not to the other calculation 
modules, we uncheck all object special properties for the dynamic shapes, in 
the object common properties.

Then, we still have to configure the dynamic shapes as dynamic and respondable. 
We do this in the shape dynamics properties. Select first the base dynamic 
shape\note{i.e. \file{robot\_dyn}}, then check the \menu{\checkbox\ Body is 
respondable}\ item. Enable the first 4 Local respondable mask flags, and 
disable the last 4 Local respondable mask flags: it is important for 
consecutive respondable links not to collide with each other. For the first 
mobile dynamic link in our robot\note{i.e. \file{robot\_link\_dyn1}}, 
we also enable the 
Body is respondable item, but this time we disable the first 4 Local 
respondable mask flags, and enable the last 4 Local respondable mask flags. 
We repeat the above procedure with all other dynamic links, while always 
alternating the Local respondable mask flags: once the model will be defined,
consecutive dynamic shapes of the robot will not generate any collision
response when interacting with each other. Try to always end up with a
construction where the dynamic base of the robot, and the dynamic last link 
of the robot have only the first 4 Local respondable mask flags enabled, so 
that we can attach the robot to a mobile platform, or attach a gripper to the
last dynamic link of the robot without dynamic collision interferences.

Finally, we still need to tag our dynamic shapes as Body is dynamic. We do 
this also in the shape dynamics properties. We can then enter the mass and 
inertia tensor properties manually, or have those values automatically 
computed\note{recommended}\ by clicking Compute mass \& inertia properties for 
selected convex shapes. Remember also this and that dynamic design 
considerations. This dynamic base of the robot is a special case: most of the
time we want the base of the robot\note{i.e. robot\_dyn}\ to be
non-dynamic\note{i.e. static}, otherwise, if used alone, the robot might 
fall during movement. But 
as soon as we attach the base of the robot to a mobile platform, we want the 
base to become dynamic\note{i.e. non-static}. We do this by enabling the Set to 
dynamic if gets parent item, then disabling the Body is dynamic item. Now run 
the simulation: all dynamic shapes, except for the base of the robot, should 
fall. That attached visual shapes will follow their dynamic pendants.

\subsubsection{Model definition}

Now we are ready to define our model. We start by building the model herarchy: 
we attach the last dynamic robot link (robot\_link\_dyn6) to its corresponding 
joint (robot\_joint6) by selecting robot\_link\_dyn6, then control-selecting
robot\_joint6, then
\menu{Menu>Edit>Make last selected object parent}. 
We could also have done this step by simply dragging object robot\_link\_dyn6 
onto robot\_link6 in the scene hierarchy. We go on by now attaching robot\_joint6
to robot\_link\_dyn5, and so on, until arrived at the base of the robot. We now
have following scene hierarchy:

\fig{Robot model hierarchy}{fig/model/19.jpg}{height=.5\textheight}

It is nice and more logical to have a simple name for the model base, since 
the model base will also represent the model itself. So we rename robot to 
robot\_visibleBase, and robot\_dyn to robot. Now we select the base of the 
hierarchy tree (i.e. object robot) and in the object common properties we 
enable Object is model base. We also enable Object/model can transfer or 
accept DNA. A model bounding box appeared, encompassing the whole robot. The 
bounding box however appears to be too large: this is because the bounding 
box also encompasses the invisible items, such as the joints. We now exclude 
the joints from the model bounding box by enabling the Don't show as inside 
model selection item for all joints. We could do the same procedure for all 
invisible items in our model. This is also a useful option in order to also 
exclude large sensors or other items from the model bounding box. We now have 
following situation:

\fig{\\Robot model bounding box}{fig/model/20.jpg}{height=.5\textheight}

We now protect our model from accidental modification. We select all visible 
objects in the robot, then enable Select base of model instead: if we now 
click a visible link in the scene, the base of the robot will be selected 
instead. This allows us to manipulate the model as if it was a single object. 
We can still select visible objects in the robot via control-shift-clicking 
in the scene, or by selecting the object in the scene hierarchy. We now put
the robot into a correct default position/orientation. First, we save current 
scene as a reference (e.g. if at a later stage we need to import CAD data 
that have the same orientation at the curent robot). Then we select the model 
and modify its position/orientation appropriately. It is considered good
practice to position the model (i.e. its base object) at X=0 and Y=0.

\fig{Robot model in default configuration}{fig/model/21.jpg}{height=.5\textheight}

We now run the simulation: the robot will collapse, since the joints are not 
controlled by default. When we added the joints in the previous stage, we 
created joints in force/torque mode, but their motor or controller was 
disabled (by default). We can now adjust our joints to our requirements. In
our case, we want a simple PID controller for each one of them. In the joint
dynamic properties, we click Motor enabled and adjust the maximum torque. We
then click Control loop enabled and select Position control (PID). We now run
the simulation again: the robot should hold its position. Try to switch the 
current physics engine to see if the behaviour is consistent across all
supported physics engines. You can do this via the appropriate toolbar button,
or in the general dynamics properties.

During simulation, we now verify the scene dynamic content via the Dynamic 
content visualization \& verification toolbar button. Now, only items that 
are taken into account by the physics engine will be display, and the display
is color-coded. It is very important to always do this, and specially when 
your dynamic model doesn't behave as expected, in order to quickly debug the
model. Similarly, always look at the scene hierarchy during simulation:
dynamically enabled objects should display a ball-bounding icon on the
right-hand side of their name.

\fig{Dynamic content visualization \& verification}{fig/model/22.jpg}{height=.5\textheight}

Finally, we need to prepare the robot so that we can easily attach a gripper 
to it, or easily attach the robot to a mobile platform (for instance). Two 
dynamically enabled shapes can be rigidly attached to each other in two 
different ways:
\begin{description}
\item{\textbf{by grouping them}}
select the shapes, then \menu{Menu>Edit>Grouping/Merging>Group selected shapes}.
\item{\textbf{by attaching them via a force/torque sensor}}
a force torque sensor can also act as a rigid link between two separate 
dynamically enabled shapes.
\end{description}

In our case, only option 2 is of interest. We create a force/torque sensor 
with \menu{Menu>Add>Force sensor}, then move it to the tip of the robot, 
then attach it to object robot\_link\_dyn6. We change its size and visual 
appearance appropriately\note{a red force/torque sensor is often perceived as an 
optional attachment point, check the various robot models available}. We also 
change its name to robot\_attachment:

\fig{Attachment force/torque sensor}{fig/model/23.jpg}{height=.5\textheight}

Now we drag a gripper model into the scene, keep it selected, then 
\keys{Ctrl+\lms}\ the attachment force sensor, then click the 
Assembling/disassembling toolbar button. The gripper goes into place:

\fig{Attached gripper}{fig/model/24.jpg}{height=.5\textheight}

The gripper knew how to attach itself because it was appropriately configured 
during its model definition. We now also need to properly configure the robot 
model, so that it will know how to attach itself to a mobile base for instance.
We select the robot model, then click Assembling in the object common 
properties. Disable Object can have the 'parent' role, then click Set matrix. 
This will memorize the current base object's local transformation matrix, and 
use it to position/orient itself relative to the mobile robot's attachment
point. To verify that we did things right, we drag the model
Models/robots/mobile/KUKA Omnirob.ttm into the scene. Then we select our 
robot model, then control-click one of the attachment points on the mobile 
platform, then click the Assembling/disassembling toolbar button. Our robot
should correctly place itself on top of the mobile robot:

\fig{Attached robot}{fig/model/25.jpg}{height=.5\textheight}

Now we could add additional items to our robot, such as sensors for instance. 
At some point we might also want to attach embedded scripts to our model, in 
order to control its behaviour or configure it for various purposes. In that 
case, make sure to understand how object handles are accessed from embedded 
scripts. We can also control/access/interface our model from a plugin, from 
a remote API client, from a ROS node, or from an add-on.

Now we make sure we have reverted the changes done during robot and gripper 
attachment, we collapse the hierarchy tree of our robot model, select the 
base of our model, then save it with \menu{Menu>File>Save model as\ldots}. 
If we saved it in the model folder, then the model will be available 
in the model brower.

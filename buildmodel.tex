\secrel{Building a clean model}\label{buildmodel}

\cp{\url{http://www.coppeliarobotics.com/helpFiles/en/buildingAModelTutorial.htm}}

This tutorial will guide you step-by-step into building a clean simulation 
model, of a robot, or any other item. This is a very important topic, maybe 
the most important aspect, in order to have a nice looking, fast displaying, 
fast simulating and stable simulation model.

To illustrate the model building process, we will be building following 
manipulator:

\fig{\ Model of robotic manipulator}{fig/model/01.jpg}{height=0.5\textheight}

\subsubsection{Building the visible shapes}

When building a new model, first, we handle only the visual aspect of it: the 
dynamic aspect\note{its undelying even more simplified/optimized model}, joints, 
sensors, etc. will be handled at a later stage.

We could now directly create primitive shapes in \vrep\ with 
\menu{Menu>Add>Primitive shape>\ldots}. When doing this, we have the option to
create pure shapes, or regular shapes. Pure shape will be optimized for 
dynamic interaction, and also directly be dynamically enabled\note{i.e. fall,
collide, but this can be disabled at a later stage}. Primitive shapes will be 
simple meshes, which might not contain enough details or geometric accuracy 
for our application. Our other option in that case would be to import a mesh 
from an external application.

When importing CAD data from an external application, the most important is to 
make sure that the CAD model is not too heavy, i.e. doesn't contain too many 
triangles. This requirement is important since a heavy model will be slow in 
display, and also slow down various calculation modules that might be used at 
a later stage\note{e.g. minimum distance calculation, or dynamics}. Following 
example is typically a no-go\note{even if, as we will see later, there are means 
to simplify the data within \vrep}:

\fig{\\Complex CAD data (in solid and wireframe)}{fig/model/02.jpg}{height=0.5\textheight}

Above CAD data is very heavy: it contains many triangles\note{more than 
47'000}, which would be ok if we would just use a single instance of it in an 
empty scene. But most of the time you will want to simulate several instances 
of a same robot, attach various types of grippers, and maybe have those robots
interact with other robots, devices, or the environment. In that case, a 
simulation scene can quickly become too slow. Generally, we recommend to model 
a robot with no more than a total of 20'000 triangles, but most of the time 
5'000..10'000 triangles would just do fine as well. Remember: less is better,
in almost every aspect.

What makes above model so heavy? First, models that contain holes and small 
details will require much more triangular faces for a correct representation. 
So, if possible, try to remove all the holes, screws, the inside of objects,
etc. from your original model data. If you have the original model data
represented as parametric surfaces/objects, then it is most of the time a 
simple matter of selecting the items and deleting them\note{e.g. in 
SolidWorks}. The second important step is to export the original data with a 
limited precision: most CAD applications let you specify the level of details 
of exported meshes. It might also be important to export the objects in 
several steps, when the drawing consists of large and small objects; this is
to avoid having large objects too precisely defined\note{too many triangles}\
and small objects too roughly defined\note{too few triangles}: simply export 
large objects first\note{by adjusting the desired precision settings}, then 
small objects\note{by adjusting up precision settings}.

\vrep\ supports currently following CAD data formats: \file{OBJ}, \file{STL}, 
\file{DXF}, \file{3DS}\note{\win\ only}, and \file{Collada}. \file{URDF}\ is 
also supported, but not mentionned here since it is not a pure mesh-based
file format.

Now suppose that we have applied all possible simplifications as described in 
previous section. We might still end-up with a too heavy mesh after import:

\fig{Imported CAD data}{fig/model/03.jpg}{height=0.33\textheight}

You can notice that the whole robot was imported as a single mesh. We will 
see later how to divide it appropriately. Notice also the wrong orientation of
the imported mesh: best is to keep the orientation as it is, until the whole
model was built, since, if at a later stage we want to import other items that
are related to that same robot, they will automatically have the correct
position/orientation relative to the original mesh.

At this stage, we have several functions at our disposal, to simplify the mesh:
\begin{description}
	\item{\textbf{Automatic mesh division}}
	allows to generate a new shape for all 
	elements that are not linked together via a common edge. This does not 
	always work for the selected mesh, but is always worth a try, since 
	working on mesh elements gives us more control than if we had to work on 
	all elements at the same time. The function can be accessed with
	\menu{Menu>Edit>Grouping/Merging>Divide selected shapes}. Sometimes, a 
	mesh will be divided more than expected. In that case, simply merge 
	elements that logically belong together\note{i.e. that will have the same
	visual attributes and that are part of the same link}\ back into one 
	single shape \menu{Menu>Edit>Grouping/Merging>Merge selected shapes}.

	\item{\textbf{Extract the convex hull}} 
	allows to simplify the mesh by transforming
	it into a convex hull. The function can be accessed with
	\menu{Menu>Edit>Morph selection into convex shapes}.

	\item{\textbf{Decimate the mesh}}
	allows to reduce the number of triangles contained
	in the mesh. The function can be accessed with
	\menu{Menu>Edit>Decimate selected shape...}.

	\item{\textbf{Remove the inside of the mesh}}
	allows to simplify the mesh by 
	removing its inside. This function is based on vision sensors and might 
	give more or less satisfying results depending on the selected settings. 
	The function can be accessed with
	\menu{Menu>Edit>Extract inside of selected shape}.
\end{description}

There is no predefined order in which above functions can/should be 
applied\note{except for the first item in the list, which should always be
tried first}, it heavily depends on the geometry of the mesh we are trying to 
simplify. Following image illustrates above functions applied to the imported 
mesh\note{let's suppose the first item in the list didn't work for us}:

\fig{Convex hull, decimated mesh, and extracted inside}{fig/model/04.jpg}{height=0.95\textheight}\pagebreak

Notice how the convex hull doesn't help us at this stage. We decide to use the 
mesh decimation function first, and run the function twice in order to divide 
the number of triangles by a total of 50. Once that is done, we extract the 
inside of the simplified shape and discard it. We end-up with a mesh containing
a total of 2'660 triangles\note{the original imported mesh contained more than 
136'000 triangles\,!}. The number of triangles/vertices a shape contains can 
be seen in the shape geometry dialog. 2'660 triangles are extremely few 
triangles for a whole robot model, and the visual appearance might suffer a
little bit from it.

At this stage we can start to divide the robot into separate 
links\note{remember, we currently have only a single shape for the whole 
robot}. You can do this in two different ways:

\begin{description}
	\item{\textbf{Automatic mesh division}}
	this function, which was already described in previous section, will 
	inspect the shape and generate a new shape for all elements that are not 
	linked together via a common edge. This does not always work, but is 
	always worth a try. The function can be accessed with
	\menu{Menu>Edit>Grouping/merging>Divide selected shapes}.
	\item{\textbf{Manual mesh division}}
	via the the triangle edit mode, you can manually select the triangles 
	than logically belong together, then click \keys{Extract shape}. This will
	generate a new shape in the scene. Delete the selected triangles after 
	that operation.
\end{description}

In the case of our mesh, method 1 worked fine:

\fig{\\Divided mesh}{fig/model/05.jpg}{height=0.65\textheight}

Now, we could further refine/simplify individual shapes. Sometimes also, a 
shape might look better if its convex hull is used instead. Othertimes, you 
will have to use several of above's described techniques iteratively, in 
order to obtain the desired result. Take for instance following mesh:

\fig{Imported mesh}{fig/model/06.jpg}{height=0.4\textheight}

The problem with above's shape is that we cannot simplify it nicely, because 
of the holes it contains. So we have to go the more complicated way via the 
shape edit mode, where we can extract individual elements that logically 
belong to the same convex sub-entity. This process can take several iterations: 
we first extract 3 approximate convex elements. For now, we ignore the 
triangles that are part of the two holes. While editing a shape in the shape 
edit mode, it can be convenient to switch the visibility layers, in order to 
see what is covered by other scene items.

\fig{Step 1}{fig/model/07.jpg}{height=0.4\textheight}

We end up with a toal of three shapes, but two of them will need further 
improvement. Now we can erase the triangles that are part of the holes. 
Finally, we extract the convex hull individually for the 3 shapes, then merge 
them back together with \menu{Menu>Edit>Grouping/Merging>merge selected shapes}:

\fig{Step 2}{fig/model/08.jpg}{height=0.6\textheight}

In \vrep, we can enable/disable edge display for each shape. We can also 
specify an angle that will be taken into account for the edge display. A
similar parameter is the shading angle, that dictates how facetted the shape 
will display. Those parameters, and a few others such as the shape color, can
be adjusted in the shape properties. Remember that shapes come in various 
flavours. In this tutorial we have only dealt with simple shapes up to now: 
a simple shape has a single set of visual attributes\note{i.e. one color, one 
shading angle, etc.}. If you merge two shapes, then the result will be a
simple shape. You can also group shapes, in which case, each shape will 
retain its visual attributes.

In next step, we can merge elements that logically belong together\note{if they
are part of the same rigid element, and if they have the same visual attributes}.
Then we change the visual attributes of the various elements. The easiest ist 
to adjust a few shapes that have different colors and visual attributes, and 
if we name the color with a specific string, we can later easily 
programmatically change that color, also if the shape is part of a compound 
shape. Then, we select all the shapes that have the same visual attributes, 
then control-select the shape that was already adjusted, then click Apply to
selection, once for the Colors, once for the other properties, in the shape
properties: this transfers all visual attributes to the selected 
shapes\note{including the color name if you provided one}. We end up with 17 
individual shapes:

\fig{Adjusted visual attributes}{fig/model/09.jpg}{height=0.5\textheight}

Now we can group the shapes that are part of the same link with
\menu{Menu>Edit>Grouping/merging>Group selected shapes}. We end up with 7 
shapes: the base of the robot\note{or base of the robot's hierarchy tree}, 
and 6 mobile links. It is also important to correctly name your objects: you 
we do this with a double-click on the object name in the scene hierarchy. The 
base should always be the robot or model name, and the other objects should 
always contain the base object name, like: \verb|robot|\ (base), 
\verb|robot_link1|, \verb|robot_proximitySensor|, etc. By defaut, shapes will 
be assigned to visibility layer 1, but can be changed in the object common 
properties. By default, only visibility layers 1-8 are activated for the 
scene. We now have following (the model \verb|ResizableFloor_5_25|\ was
temporarily made invisible in the model properties dialog):

\fig{\\Individual elements compositn the robot}{fig/model/10.jpg}{height=0.8\textheight}

When a shape is created or modified, V-REP will automatically set its reference
frame position and orientation. A shape's reference frame will always be 
positioned at the shape's geometric center. The frame orientation will be 
selected so that the shape's bounding box remains as small as possible. This 
does not always look nice, but we can always reorient a shape's reference 
frame at any time. We now reorient the reference frames of all our created 
shapes with \menu{Menu>Edit>Reorient bounding box>with reference frame of 
world}. You have more options to reorient a reference frame in the shape 
geometry dialog.

\subsubsection{Building the joints}

Now we will take care of the joints/motors. Most of the time, we know the 
exact position and orientation of each of the joints. In that case, we simply 
add the joints with \menu{Menu>Add>Joints>\ldots}, then we can change their 
position and orientation with the coordinate and transformation dialogs. In 
other situations, we only have the Denavit-Hartenberg\note{i.e. D-H}\ parameters. 
In that case, we can build our joints via the tool model located in 
\file{Models/tools/Denavit-Hartenberg joint creator.ttm}, in the model browser.
Othertimes, we have no information about the joint locations and orientations. 
Then, we need to extract them from the imported mesh. Let's suppose this is 
our case. Instead of working on the modified, more approximate mesh, we open a 
new scene, and import the original CAD data again. Most of the time, we can 
extract meshes or primitive shapes from the original mesh. The first step is 
to subdivide the original mesh. If that does not work, we do it via the 
triangle edit mode. Let's suppose that we could divide the original mesh. We 
now have smaller objects that we can inspect. We are looking for revolute 
shapes, that could be used as reference to create joints at their locations, 
with the same orientation. First, remove all objects that are not needed. It 
is sometimes also useful to work across several opened scenes, for easier 
visualization/manipulation. In our case, we focus first on the base of the
robot: it contains a cylinder that has the correct position for the first 
joint. In the triangle edit mode, we have:

\fig{\\Robot base: normal and triangle edit mode visualization}{fig/model/11.jpg}{height=0.8\textheight}

We change the camera view via the page selector toolbar button, in order to 
look at the object from the side. The fit-to-view toolbar button can come in 
handy to correctly frame the object in edition. Then we switch to the vertex
edit mode and select all vertices that belong to the upper disc. Remember that
by switching some layers on/off, we can hide other objects in the scene. Then
we switch back to the triangle edit mode:

\fig{\\Selected upper disc, vertex edit mode (1 \& 2), triangle edit mode (3)}{fig/model/12.jpg}{height=0.8\textheight}

Now we click \menu{Extract cylinder}\note{\menu{Extract shape}\ would also work 
in that case}, this just created a cylinder shape in the scene, based on the 
selected triangles. We leave the edit mode and discard the changes. Now we add
a revolute joint with \menu{Menu>Add>Joint>Revolute}, keep it selected, then 
\keys{Ctrl-\lms}\ select the extracted cylinder shape. In the coordinate and 
transformation dialogs we click the Position/Translation button, then, in the 
Object/item position section, we click Apply to selection (the one under the 
edit boxes): this basically copies the x/y/z position of the cylinder to the 
joint. Both positions are now identical. We click the Orientation/Rotation 
button, then in the Object/item orientation section, we click Apply to 
selection: the orientation of our selected objects is now also the same. 
Sometimes, we will need to additionally rotate the joint about 90/180 degrees
around its own reference frame in order to obtain the correct orientation or
rotation direction. We could do that in the Object/item rotation opertations 
section if needed (in that case, do not forget to click the Own frame button).
In a similar way we could also shift the joint along its axis, or even do more
complex operations. This is what we have:

\fig{Joint in correct location, with the correct orientation}{fig/model/13.jpg}{height=0.5\textheight}

Now we copy the joint back into our original scene, and save it\note{do not 
forget to save your work on a regular basis\,! The undo/redo function is 
useful, but doesn't protect you against other mishaps}. We repeat above 
procedure for all the joints in our robot, then rename them. We also make all 
joints a little bit longer in the joint properties, in order to see them all. 
By defaut, joints will be assigned to visibility layer 2, but can be changed
in the object common properties. We assign now all joints to visibility layer
10, then temporarily enable visibility layer 10 for the scene to also 
visualize those joints (by default, only visibility layers 1-8 are activated 
for the scene). This is what we have\note{the model \file{ResizableFloor\_5\_25}\
was temporarily made invisible in the model properties dialog}:

\fig{\\Joints in correct configuration}{fig/model/14.jpg}{height=0.5\textheight}

At this point, we could start to build the model hierarchy and finish the 
model definition. But if we want opur robot to be dynamically enabled, then 
there is an additional intermediate step:

\subsubsection{Building the dynamic shapes}

If we want our robot to be dynamically enabled, i.e. react to collisions, fall,
etc., then we need to create/configure the shapes appropriately: a shape can be:
\begin{description}
	\item{\textbf{dynamic or static}}
	a dynamic (or non-static) shape will fall and be influences by external
	forces/torques. A static (or non-dynamic) shape on the other hand, will 
	stay in place, or follow the movement of its parent in the scene hierarchy.
	\item{\textbf{respondable or non-respondable}}
	a respondable shape will cause a collision reaction with other respondable 
	shapes. They (and/or) their collider, will be influenced in their movement 
	if they are dynamic. On the other hand, non-respondable shapes will not 
	compute a collision response if they collide with other shapes.
\end{description}



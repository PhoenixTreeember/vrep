\secly{Установка}\label{install}

\begin{description}
	\item{\vrep} \href{http://www.coppeliarobotics.com/downloads.html}{скачайте}
		и установите\note{\menu{Next>Next>Next}} версию \emph{Pro Edu}

	\item{\prog{MinGW}} для сборки плагинов и интерпретатора \bi\ нужен 
		компилятор \cpp. Сам \vrep\ собран с помощью GNU G++ (MinGW), поэтому
		\emph{написание плагинов с использованием Visual C не поддерживается}.
		
		Скачайте инсталлятор
		\href{http://www.mingw.org/}{MinGW}\ и установите пакеты
		\prog{g++}, \prog{flex}\ и \prog{bison}. 

		\menu{\file{mingw-get-setup.exe}>Install>Directory>\file{C:/MinGW}}

		\menu{\checkbox\ \ldots\ also install support for the GUI}
		\menu{\checkbox\ \ldots\ in the start menu}
		\menu{\uncheckbox\ \ldots\ on the desktop}

		\menu{Continue>Continue}

		\menu{Basic Setup}

		\menu{\checkbox\ \file{mingw32-base}>\lms>Mark for Installation}
		отладчик \prog{gdb}\ и утилита сборки проекта \prog{make}.

		\menu{\checkbox\ \file{mingw32-gcc-g++}}
		компиляторы \prog{gcc}\note{чистый Си}\ и \prog{g++}\note{\cpp},
		\prog{binutils}\note{линкер, ассемблер i386, утилиты работы с 
		объектными файлами .elf}

		\menu{All Packages>MSYS>MinGW Developer Toolkit}

		\menu{\checkbox\ \file{msys-bison.bin}}
		генератор синтаксических анализаторов \prog{bison}\note{yacc}\ref{parser}

		\menu{\checkbox\ \file{msys-flex.bin}}
		генератор лексеров \prog{flex}\note{lex}\ref{lexer}

		\menu{Installation>Apply Changes>Apply>Close}

Для запуска утилит нужно добавить пути поиска в системную переменную \verb|PATH|

\menu{\winstart>Компьютер>\rms>Свойства>Дополнительные параметры>Переменные среды}

\menu{Переменные среды пользователя...}

\verb|PATH = C:\MinGW\msys\1.0\bin;C:\MinGW\bin;...|

\item{\prog{LLVM}} фреймворк для разработки компиляторов, трансляторов,
статических анализаторов и оптимизаторов кода, а \prog{clang}\ --- компилятор
Си/\cpp\ на его основе. Если вы хотите ускорить интерпретатор вашего скриптового
языка, добавив в него динамическую компиляция в машинный код, установка, сборка
и установка LLVM подробно рассмотрена в \cite{lexman}.

\end{description}

\subsubsection{Обновление}

\begin{description}

\item{\prog{MinGW}}

\menu{\winstart>Все программы>MinGW Installation Manager}

\menu{Installation>Update Catalogue>Close}

\menu{Installation>Mark All Upgrades}

\menu{Installation>Apply Changes}

\end{description}

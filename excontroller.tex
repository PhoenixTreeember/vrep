\secrel{Внешние контроллеры}\label{excontroller}\secdown

\cp{http://www.coppeliarobotics.com/helpFiles/en/externalControllerTutorial.htm}

There are several ways one can control a robot or simulation in \vrep:

\begin{itemize}

\item The most convenient way is to write a child script that will handle the
	behaviour of a given robot or model. It is the most convenient way, 
	because child scripts are directly attached to scene objects, they will 
	be duplicated together with their associated scene objects, they do not 
	need any compilation with an external tool, they can run in threaded or 
	non-threaded mode, they can be extended via custom Lua function or via 
	a Lua extension library. Another major advantage in using child scripts:
	there is no communication lag as with the last 3 methods mentioned in 
	this section (i.e. the regular API is used), and child scripts are part
	of the application main thread (inherent synchronous operation). There 
	are several drawback to writing scripts however: you don't have the 
	choice of the programming language, you can't have the fastest code, 
	and you can't directly access external function libraries, except the
	Lua extension libraries.

\item Another way one can control a robot or a simulation is by writing a 
	plugin. The plugin mechanism allows for callback mechanisms, custom Lua
	function registration, and of course access to external function 
	libraries. A plugin is often used in conjunction with child scripts 
	(e.g. the plugin registers custom Lua functions, that, when called from 
	a child script, will call back a specific plugin function). A major
	advantage in using plugins is also that there is no communication lag 
	as with the last 3 methods mentioned in this section (i.e. the regular 
	API is used), and that a plugin is part of the application main thread 
	(inherent synchronous operation). The drawbacks with plugins are: they
	are more complicated to program, and they need to be compiled with an
	external too. Refer also to the plugin tutorial.

\item A third way one can control a robot or a simulation is by writing an 
	external client application that relies on the remote API. This is a 
	very convenient and easy way, if you need to run the control code from
	an external application, from a robot or from another computer. This 
	also allows you to control a simulation or a model (e.g. a virtual 
	robot) with the exact same code as the one that runs the real robot. 
	The remote API functionality relies on the remote API plugin (on the 
	server side), and the remote API code on the client side. Both 
	programs/projects are open source (i.e. can be easily extended or
	translated for support of other languages) and can be found in the 
	'programming' directory of V-REP's installation. Currently there are 
	seven supported languages: C/C++, Python, Java, Matlab, Octave, 
	Lua and Urbi.

\item A forth way to control a robot or a simulation is via a ROS node. 
	In a similar way as the remote API, ROS is a convenient way to have 
	several distributed processes communicate with each other. While the 
	remote API is very lightweight and fast, it allows only communication 
	with V-REP. ROS on the other hand allows connecting virtually any 
	number of processes with each other, and a large amount of compatible
	libraries are available. It is however heavier and more complicated
	than the remote API. Refer to the ROS interface for details.

\item A fifth way to control a robot or a simulation is by writing an 
	external application that communicates via various means (e.g. pipes, 
	sockets, serial port, etc.) with a V-REP plugin or V-REP script. Two 
	major advantages are the choice of programming language, which can be 
	just any language, and the flexibility. Here also, the control code 
	can run on a robot, or a different computer. This way of controlling 
	a simulation or a model is however more tedious that the method with
	the remote API.

\end{itemize}

The V-REP scene file related to this tutorial is "controlTypeExamples.ttt" 
located in V-REP's "scenes" folder. Open the scene and run the simulation:

\figx{fig/excontrol/01.jpg}{height=.5\textheight}

The simulation illustrates the 5 control types previously mentioned: the 
orange robot is controlled by an embedded child script, the blue robot is 
controlled by a plugin, the purple robot is controlled via the remote API 
and an external client application, the red robot is controlled via an 
external ROS application, and the green robot is controlled by an external 
server application via a custom socket communication. In all 5 cases, child 
scripts are used, mainly to make the link with the outside world (e.g. 
launch the correct client application, and pass the correct object handles 
to it). There are two other ways one can control a robot, a simulation, or
the simulator itself: by using customization scripts, or add-ons. They are
however not recommended for control and should be rather used to handle
functionality while simulation is not running.

As an example, the child script linked to the purple robot has following 
main tasks:
\begin{itemize}[nosep]
	\item Search for a free socket connection port
	\item Start a remote API server service on the above port
	\item Launch the controller application ("bubbleRobClient") with the
		chosen connection port and some object handles as argument
\end{itemize}

As another example, the child script linked to the red robot has following 
main tasks:
\begin{itemize}[nosep]
	\item Check if the ROS plugin for V-REP was loaded
	\item Launch the controller application ("rosBubbleRob") with some object
		handles as arguments
\end{itemize}

Yet, as another example, the child script linked to the green robot has 
following main tasks:
\begin{itemize}[nosep]
	\item Search for a free socket connection port
	\item Launch the controller application ("bubbleRobServer") with the 
		chosen connection port as argument
	\item Locally connect to the controller application
	\item At each simulation pass, send the sensor values to the controller, 
		and read the desired motor values from the controller
	\item At each simulation pass, apply the desired motor values to
		the robot's joints
\end{itemize}

The project files for the "bubbleRobClient", "rosBubbleRob", and
"bubbleRobServer" applications are located in V-REP's installation 
folder "programming".

Run the simulation, and copy-and-paste all robots: you will see that the 
duplicated robots will directly be operational, since their attached child 
scripts are in charge of launching new instances of "bubbleRobClient", 
"rosBubbleRob", "bubbleRobServer", or calling the appropriate plugin Lua
extension function.

Finally, notice that the green robot is controlled via "bubbleRobServer", 
but that its child script acts as client that communicates with the server. 
You could also imagine having the client part not inside of a child script, 
but inside of a plugin. Such an example is illustrated in the robot 
language interpreter integration tutorial \ref{langtutor}.

\secrel{Child control script}\label{chscript}

\secrel{Плагин-контроллер}\label{ctlplugin}


\secrel{Remote API}\label{remoteapi}

\secrel{Узел ROS}\label{rosnode}

\secrel{Сетевые сокеты TCP/IP}\label{sockets}


\secup

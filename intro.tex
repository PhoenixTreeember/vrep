\secly{Введение}\label{intro}

\cp{\cite{habr57}}

Программирование роботов\ --- это интересно.

Многие наверное видели японских гуманоидных роботов, или французский учебный
робот \href{https://en.wikipedia.org/wiki/Nao_(robot)}{NAO}, 
интересным выглядит проект обучаемого робота-манипулятор 
\href{http://www.rethinkrobotics.com/baxter/}{Baxter}. 
Промышленные манипуляторы 
\href{http://www.kuka-robotics.com/russia/ru/}{KUKA}\ из Германии\ --- это 
классика. Кто-то программирует системы конвейерной обработки (фильтрации, 
сортировки). Дельта роботы. Есть целый пласт\ --- управление 
квадрокоптером/алгоритмы стабилизации. И конечно же 
простые трудяги на складе — Line Follower. \textit{И самый доступный для 
самостоятельного изготовления, и в то же время полезный 
вариант\ --- \textbf{\emph{универсальные
роботележки}} на колесном или гусеничном ходу, с размерами от настольного
микробота размером с мышку, до танков и карьерных робоэлектровозов}.

\noindent
\fig{NAO}{fig/NAO.jpg}{height=0.3\textheight}
\fig{Вахтёр}{fig/Baxter.jpg}{height=0.3\textheight}
\fig{\ KUKA}{fig/KUKA.jpg}{height=0.3\textheight}
\bigskip

Но всё это как правило\ --- не дешевые игрушки, поэтому доступ к роботам есть в
специализированных лабораториях или институтах/школах где получили 
финансирование и есть эти направления. Всем же остальным разработчикам (кому
интересна робототехника)\ --- остаётся завистливо смотреть.

Некоторое время назад я вышел на достаточно интересную систему\ --- 3D
робосимулятор \vrep, от швейцарской компании Coppelia Robotics.

К своему (приятному) удивлению я обнаружил, что эта система:
\begin{itemize}[nosep]
\item имеет большой функционал\note{система разрабатывается с марта 2010 года}
\item полностью open-source\note{выложена в открытый доступ в 2013 году}
\item кроссплатформенная\ --- \win, mac, \lin\note{рализована на фреймворке Qt}
\item имеет API и библиотеки для работы с роботами через 
	C/\cpp, \py, Java, Lua, Matlab, Octave или Urbi
\item бесплатная для некоммерческого использования
\end{itemize}

Все объекты, которые программируются в этой системе\ --- "живут"\ в реальном с 
точки зрения физических законов мире\ --- есть гравитация, можно захватывать 
предметы, столкновения, датчики расстояния, видео датчики и т.п.

Поработав некоторое время с этой системой, я решил рассказать про неё 
читателям хабра.

Да, и на картинке скриншот из \vrep, и модели роботов\ --- которые вы можете 
программировать, и смотреть поведение, прямо на вашем компьютере.

\figx{fig/sshot.jpg}{height=0.5\textheight}

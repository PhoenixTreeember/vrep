\secrel{Динамика}\label{dynamics}

\cp{\url{http://www.coppeliarobotics.com/helpFiles/en/dynamicsModule.htm}}

Модуль динамической симуляции \vrep\ поддерживает несколько библиотек:
\begin{enumerate}[nosep]
\item \href{http://www.bulletphysics.org/}{Bullet physics library},
open-source физический движок, поддерживающий 3D столкновения, динамику
жесткого тела, динамику эластичного тела\note{этот функционал не поддерживается
в \vrep}. Этот движок используется в играх и рализации спецэффектов в фильмах,
поэтому он считается \emph{игровым физическим движком}\note{с низкой точностью}.
	
\item \href{http://www.ode.org/}{Open Dynamics Engine},
open source физический движок из двух основных компонентов: динамика твердого 
тела и определение столкновений. Он используется во многих приложениях и играх
как \emph{игровой движок}.

\item \href{http://www.vxsim.com/}{Vortex Dynamics}
закрытый \emph{коммерческий}\ физический двжок дающий высокую точность 
симуляции. Vortex поддерживает реальные физические параметры\note{соответствующие
физическим единицам измерения}\ для широкого набора приложений,
делая симуляцию реалистичной и точной. Vortex в основном используется
в быстрых/точных применениях в промышленных и исследоваельких целях.
Vortex в поставке \vrep\ --- коммерческий продукт, и \emph{бесплатная версия
поддерживают симуляцию только в течение \textbf{20 секунд}}.

\item \href{http://newtondynamics.com/forum/newton.php}{Newton Dynamics}
кроссплатформенная реалистичная библиотека физической симуляции.
Она реализует детерминированный решатель, не использующий традиционные
LCP или итерационные методы, и обладает одновременно сходимостью и скоростью.
Это делает Newton Dynamics не только игровым движком, но и средством
точной физической симуляции. Текущая версия плагина является бета-версией, и
\emph{не работает в \win\ сборке\ --- \vrep\ падает при попытке использования}
	
\end{enumerate}

Пользователь может свободно и быстро переключаться между физическими движками
в зависимости от текущих условий. Причина такого разнообразия физических 
движков заключается в том, что физическая симуляция является сложной задачей,
которая модет решаться с различной точностью, скоростью, или поддержкой
особых функций.

Динамический модуль \vrep\ позволяет симулировать взаимодействие между объектами
близко к реальности. Они могут падать, сталкиваться, отскакивать, при симуляции
манипуляторов симулируется сила трения при захвате, 
конвеерные ленты могут перемещать объекты с учетом скольжения, а подвижные
платформы движутся реалистично по неровным поверхностям.

\fig{\\Иллюстрация симуляции динамики}{fig/dynamics.jpg}{height=0.6\textheight}

В отличие от многих других пакетов симуляции, \vrep\ не только чисто 
динамический симулятор. Он может рассматриваться как гибридный, сочетая
кинематику и динамику для получения наилучшей производительности в зависимости
от сценария использования. Сейчас физические движки все еще основаны на очень
большом числе упрощений и отностельно неточны и медленны, поэтому для достижения
наилучших результатов пытайтесь пользоваться кинематическая 
симуляцией\note{например для робототехнических манипуляторов},
и динамической если кинематика не работает\note{например захват робота}. 
Если ваша симуляция мобильного робота не предусматривает столкновений
или физического взаимодействия со средой\note{большинство мобильных роботов
не часто этого требуют}, и робот движется исклчительно по плоской 
поверхности\note{случай для большинства применений мобильных работов},
попытайтесь использовать кинематические или геометрические вычисления для
симуляции движения. Результат будет быстрее и точнее.

Некоторые данные, получаемые при работе модуля динамики, могут быть записаны
объектами-графиками. См. графики и типы графических потоков данных\ \ref{graph}\
для получения более подробной информации по записи динамических данных.

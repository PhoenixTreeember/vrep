\secrel{BubbleRob}\label{bublerob}

\cp{\url{http://www.coppeliarobotics.com/helpFiles/en/bubbleRobTutorial.htm}}

This tutorial will try to introduce quite many \vrep\ functionalities while 
designing the simple mobile robot \file{BubbleRob}. 

The \vrep\ scene file 
related to this tutorial is located in \vrep's installation folder's 
\file{tutorials/BubbleRob}\ folder. Following figure illustrates the 
simulation scene that we will design:

\figx{fig/bubble/01.jpg}{height=0.5\textheight}

Since this tutorial will fly over many different aspects, make sure to also 
have a look at the other tutorials, mainly the tutorial about building a 
simulation model\ \ref{buildmodel}. First of all, freshly start \vrep. 
The simulator displays 
a default scene. We will start with the body of \file{BubbleRob}.

We add a primitive sphere of diameter 0.2 to the scene with
\menu{Menu bar>Add>Primitive shape>Sphere}. We adjust the X-size item to 0.2, 
then click \keys{OK}. The created sphere will appear in the visibility layer 
1 by default, and be dynamic and respondable\note{since we kept the item 
\checkbox\ Create dynamic and respondable shape enabled}. This means that 
BubbleRob's body will be falling and able to react to collisions with other 
respondable shapes\note{i.e. simulated by the physics engine}. We can see 
this is the shape dynamics properties: items Body is respondable and 
\checkbox\ Body is dynamic are enabled. We start the simulation\note{via the 
toolbar button, or by pressing \keys{Ctrl+Space}\ in the scene window}, and 
copy-and-paste the created sphere with
\menu{Menu bar>Edit>Copy selected objects}\ then \menu{Edit>Paste buffer}, or 
with \keys{Ctrl+C}\ then \keys{Ctrl+V}: the two spheres will react to 
collision and roll away. We stop the simulation: the duplicated sphere will 
automatically be removed. This default behaviour can be modified in the 
simulation dialog.

\figx{fig/bubble/02.jpg}{height=0.5\textheight}

\figx{fig/bubble/03.jpg}{height=0.5\textheight}

\figx{fig/bubble/04.jpg}{height=0.5\textheight}

\figx{fig/bubble/05.jpg}{height=0.1\textheight}

\figx{fig/bubble/06.jpg}{height=0.1\textheight}

\figx{fig/bubble/07.jpg}{height=0.5\textheight}

\figx{fig/bubble/08.jpg}{height=0.5\textheight}

\figx{fig/bubble/09.jpg}{height=0.1\textheight}

\figx{fig/bubble/10.jpg}{height=0.1\textheight}

\figx{fig/bubble/11.jpg}{height=0.5\textheight}

\figx{fig/bubble/12.jpg}{height=0.1\textheight}

\figx{fig/bubble/13.jpg}{height=0.5\textheight}

\figx{fig/bubble/14.jpg}{height=0.1\textheight}


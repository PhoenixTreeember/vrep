\secrel{BubbleRob}\label{bublerob}

\cp{\url{http://www.coppeliarobotics.com/helpFiles/en/bubbleRobTutorial.htm}}

This tutorial will try to introduce quite many \vrep\ functionalities while 
designing the simple mobile robot \file{BubbleRob}. 

The \vrep\ scene file 
related to this tutorial is located in \vrep's installation folder's 
\file{tutorials/BubbleRob}\ folder. Following figure illustrates the 
simulation scene that we will design:

\figx{fig/bubble/01.jpg}{height=0.5\textheight}

Since this tutorial will fly over many different aspects, make sure to also 
have a look at the other tutorials, mainly the tutorial about building a 
simulation model\ \ref{buildmodel}. First of all, freshly start \vrep. 
The simulator displays 
a default scene. We will start with the body of \file{BubbleRob}.

We add a primitive sphere of diameter 0.2 to the scene with
\menu{Menu bar>Add>Primitive shape>Sphere}. We adjust the X-size item to 0.2, 
then click \keys{OK}. The created sphere will appear in the visibility layer 
1 by default, and be dynamic and respondable\note{since we kept the item 
\checkbox\ Create dynamic and respondable shape enabled}. This means that 
BubbleRob's body will be falling and able to react to collisions with other 
respondable shapes\note{i.e. simulated by the physics engine}. We can see 
this is the shape dynamics properties: items Body is respondable and 
\checkbox\ Body is dynamic are enabled. We start the simulation\note{via the 
toolbar button, or by pressing \keys{Ctrl+Space}\ in the scene window}, and 
copy-and-paste the created sphere with
\menu{Menu bar>Edit>Copy selected objects}\ then \menu{Edit>Paste buffer}, or 
with \keys{Ctrl+C}\ then \keys{Ctrl+V}: the two spheres will react to 
collision and roll away. We stop the simulation: the duplicated sphere will 
automatically be removed. This default behaviour can be modified in the 
simulation dialog.

We also want the BubbleRob's body to by usable by the other calculation 
modules (e.g. the minimum distance calculation module). For that reason, we
enable Collidable, Measurable, Renderable and Detectable in the object 
common properties for that shape, if not already enabled. If we wanted, we 
could now also change the visual appearance of our sphere in the shape 
properties.

Now we open the position and translation dialog, select the sphere 
representing BubbleRob's body, and in the dialog's section Object/item 
translation \& position scaling operations, we enter 0.02 for Along Z. We 
make sure that the Relative to-item is set to World. Then we click Translate
selection. This translates all selected objects by 2\,cm along the absolute 
Z-axis, and effectively lifted our sphere a little bit. In the scene 
hierarchy, we double-click the sphere's name, so that we can edit its name.
We enter bubbleRob and press enter.

Next we will add a proximity sensor so that BubbleRob knows when it is 
approaching obstacles: we select
\menu{Menu>Add>Proximity sensor>Cone type}. 
In the orientation and rotation dialog, in section Object/item 
rotation operations, we enter 90 for Around Y and for Around Z, then click
Rotate selection. In the position and translation dialog, in section 
Object/item position, we enter 0.1 for X-coord. and 0.12 for Z-coord. The 
proximity sensor is now correctly positioned relative to BubbleRob's body. 
We double-click the proximity sensor's icon in the scene hierarchy to open
its properties dialog. We click Show volume parameter to open the proximity
sensor volume dialog. We adjust items Offset to 0.005, Angle to 30 and Range
to 0.15. Then, in the proximity sensor properties, we click Show detection
parameters. This opens the proximity sensor detection parameter dialog. We
uncheck item Don't allow detections if distance smaller than then close that
dialog again. In the scene hierarchy, we double-click the proximity sensor's 
name, so that we can edit its name. We enter bubbleRob\_sensingNose 
and press enter.

We select bubbleRob\_sensingNose, then control-select bubbleRob, then click
\menu{Menu>Edit>Make last selected object parent}. This attaches the sensor 
to the body of the robot. We could also have dragged bubbleRob\_sensingNose 
onto bubbleRob in the scene hierarchy. This is what we now have:

\fig{\\Proximity sensor attached to bubbleRob's body}{fig/bubble/02.jpg}{height=0.5\textheight}

Next we will take care of BubbleRob's wheels. We create a new scene with
\menu{Menu>File>New scene}. It is often very convenient to work across 
several scenes, in order to visualize and work only on specific elements. 
We add a pure primitive cylinder with dimensions (0.08,0.08,0.02). As for 
the body of BubbleRob, we enable Collidable, Measurable, Renderable and 
Detectable in the object common properties for that cylinder, if not already
enabled. Then we set the cylinder's absolute position to (0.05,0.1,0.04) 
and its absolute orientation to (-90,0,0) (we can do this in the Coordinates
and transformations dialog). We change the name to bubbleRob\_leftWheel. We
copy and paste the wheel, and set the absolute Y coordinate of the copy to
-0.1. We rename the copy to bubbleRob\_rightWheel. We select the two wheels,
copy them, then switch back to scene 1, then paste the wheels.

We now need to add joints (or motors) for the wheels. We click
\menu{Menu>Add>Joint>Revolute}\ to add a revolute joint to the scene. Most
of the time, when adding a new object to the scene, the object will appear 
at the origin of the world. We Keep the joint selected, then control-select
bubbleRob\_leftWheel. In the position and translation dialog, in section
Object/item position, we click the Apply to selection button at the bottom 
of that section: this positioned the joint at the center of the left wheel.
Then, in the orientation and rotation dialog, in section Object/item 
orientation, we do the same: this oriented the joint in the same way as the
left wheel. We rename the joint to bubbleRob\_leftMotor. We now double-click
the joint's icon in the scene hierarchy to open the joint properties dialog.
Then we click Show dynamic parameters to open the joint dynamics properties
dialog. We enable the motor, and check item Lock motor when target velocity
is zero. We now repeat the same procedure for the right motor and rename it
to bubbleRob\_rightMotor. Now we attach the left wheel to the left motor, the
right wheel to the right motor, then attach the two motors to bubbleRob.
This is what we have:

\fig{\\Proximity sensor, motors and wheels}{fig/bubble/03.jpg}{height=0.5\textheight}

We run the simulation and notice that the robot is falling backwards. We are 
still missing a third contact point to the floor. We now add a small slider
(or caster). In a new scene we and add a pure primitive sphere with diameter
0.05 and make the sphere Collidable, Measurable, Renderable and Detectable
(if not already enabled), then rename it to bubbleRob\_slider. We set the
Material to noFrictionMaterial in the shape dynamics properties. To rigidly
link the slider with the rest of the robot, we add a force sensor object 
with [Menu bar --> Add --> Force sensor]. We rename it to 
bubbleRob\_connection and shift it up by 0.05. We attach the slider to the
force sensor, then copy both objects, switch back to scene 1 and paste them.
We then shift the force sensor by -0.07 along the absolute X-axis, then
attach it to the robot body. If we run the simulation now, we can notice 
that the slider is slightly moving in relation to the robot body: this is 
because both objects (i.e. bubbleRob\_slider and bubbleRob) are colliding 
with each other. To avoid strange effects during dynamics simulation, we
have to inform V-REP that both objects do not mutually collide, and we do
this in following way: in the shape dynamics properties, for bubbleRob\_slider
we set the local respondable mask to 00001111, and for bubbleRob, we set the
local respondable mask to 11110000. If we run the simulation again, we can
notice that both objects do not interfere anymore. This is what we now have:

\fig{\\Proximity sensor, motors, wheels and slider}{fig/bubble/04.jpg}{height=0.5\textheight}

We run the simulation again and notice that BubbleRob slightly moves, even 
with locked motor. We also try to run the simulation with different physics
engines: the result will be different. Stability of dynamic simulations is
tightly linked to masses and inertias of the involved non-static shapes. For
an explanation of this effect, make sure to carefully read this and that
sections. We now try to correct for that undesired effect. We select the two
wheels and the slider, and in the shape dynamics dialog we click three times
M=M*2 (for selection). The effect is that all selected shapes will have their
masses multiplied by 8. We do the same with the inertias of the 3 selected
shapes, then run the simulation again: stability has improved. In the joint
dynamics dialog, we set the Target velocity to 50 for both motors. We run
the simulation: BubbleRob now moves forward and eventually falls off the 
floor. We reset the Target velocity item to zero for both motors.

The object bubbleRob is at the base of all objects that will later form the 
BubbleRob model. We will define the model a little bit later. In the mean
time, we want to define a collection of objects that represent BubbleRob.
For that we define a collection object. We click [Menu bar --> Tools --> 
Collections] to open the collection dialog. Alternatively we can also open 
the dialog by clicking the appropriate toolbar button:

\figx{fig/bubble/05.jpg}{height=0.1\textheight}

In the collection dialog, we click Add new collection. A new collection
object appears in the list just below. For now the newly added collection 
is still empty (not defined). While the new collection item is selected in
the list, select bubbleRob in the scene hierarchy, and then click Add in 
the collection dialog. Our collection is now defined as containing all
objects of the hierarchy tree starting at the bubbleRob object (the 
collection's composition is displayed in the Composing elements and 
attributes section). To edit the collection name, we double-click it, and
rename it to bubbleRob\_collection. We close the collection dialog.

At this stage we want to be able to track the minimum distance between 
BubbleRob and any other object. For that, we open the distance dialog
with [Menu bar --> Tools --> Calculation module properties]. Alternatively
we can also open the calculation module properties dialog with the
appropriate toolbar button:

\figx{fig/bubble/06.jpg}{height=0.1\textheight}

In the distance dialog, we click Add new distance object and select a 
distance pair: [collection] bubbleRob\_collection - all other measurable 
objects in the scene. This just added a distance object that will measure 
the smallest distance between collection bubbleRob\_collection (i.e. any 
measurable object in that collection) and any other measurable object in 
the scene. We rename the distance object to bubbleRob\_distance with a 
double-click in its name. We close the distance dialog. When we now run 
the simulation, we won't see any difference, since the distance object 
will try to measure (and display) the smallest distance segment between 
BubbleRob and any other measurable object in the scene. The problem is 
that at this stage there is no other measurable object in the scene (the
shape defining the floor has its measurable property turned off by 
default). At a later stage in this tutorial, we will add obstacles
to our scene.

Next we are going to add a graph object to BubbleRob in order to display 
above smallest distance, but also BubbleRob's trajectory over time. We 
click [Menu bar --> Add --> Graph] and rename it to bubbleRob\_graph. We 
attach the graph to bubbleRob, and set the graph's absolute coordinates 
to (0,0,0.005). Now we open the graph properties dialog by double-clicking 
its icon in the scene hierarchy. We uncheck Display XYZ-planes, then click
Add new data stream to record and select Object: absolute x-position for 
the Data stream type, and bubbleRob\_graph for the Object / item to record. 
An item has appeared in the Data stream recording list. That item is a
data stream of bubbleRob\_graph's absolute x-coordinate (i.e. the
bubbleRobGraph's object absolute x position will be recorded). Now we also
want to record the y and z positions: we add those data streams in a similar
way as above. We now have 3 data streams that represent BubbleRob's x-, y-
and z-trajectories. We are going to add one more data stream so that we are
able to track the minimum distance between our robot and its environment: 
we click Add new data stream to record and select Distance: segment length 
for the Data stream type, and bubbleRob\_distance for the Object / item to 
record. In the Data stream recording list, we now rename Data to 
bubbleRob\_x\_pos, Data0 to bubbleRob\_y\_pos, Data1 to bubbleRob\_z\_pos, and 
Data2 to bubbleRob\_obstacle\_dist.

We select bubbleRob\_x\_pos in the Data Stream recording list and in the Time 
graph properties section, uncheck Visible. We do the same for 
bubbleRob\_y\_pos and bubbleRob\_z\_pos. By doing so, only the
bubbleRob\_obstacle\_dist data stream will be visible in a time graph.
Following is what we should have:

\fig{Graph properties}{fig/bubble/07.jpg}{height=0.5\textheight}

Next we will set-up a 3D curve that displays BubbleRob's trajectory: we 
click Edit 3D curves to open the XY graph and 3D curve dialog, then click 
Add new curve. In the dialog that pops open, we select bubbleRob\_x\_pos for
the X-value item, bubbleRob\_y\_pos for the Y-value item and bubbleRob\_z\_pos 
for the Z-value item. We rename the newly added curve from Curve to 
bubbleRob\_path. Finally, we check the Relative to world item and set
Curve width to 4:

\fig{3D curve properties}{fig/bubble/08.jpg}{height=0.5\textheight}

We close all dialogs related to graphs. Now we set one motor target 
velocity to 50, run the simulation, and will see BubbleRob's trajectory 
displayed in the scene. We then stop the simulation and reset the motor 
target velocity to zero.

Next, we will add a visualization window for the minimum distance data 
stream with [Menu bar menu --> Add --> Floating view]. We select 
bubbleRob\_graph, then in the floating view, right-click [Popup menu -->
View --> Associate view with selected graph]. Running the simulation will 
not yet display anything in the graph window, since there are still no 
objects (i.e. obstacles) to measure against. Let's now add some obstacles!

We add a pure primitive cylinder with following dimensions: (0.1, 0.1, 0.2).
We want this cylinder to be static (i.e. not influenced by gravity or 
collisions) but still exerting some collision responses on non-static 
respondable shapes. For this, we disable Body is dynamic in the shape 
dynamics properties. We also want our cylinder to be Collidable, 
Measurable, Renderable and Detectable. We do this in the object common 
properties. Now, while the cylinder is still selected, we click the 
object translation toolbar button:

\figx{fig/bubble/09.jpg}{height=0.1\textheight}

Now we can drag any point in the scene: the cylinder will follow the 
movement while always being constrained to keep the same Z-coordinate. We 
copy and paste the cylinder a few times, and move them to positions around
BubbleRob (it is most convenient to perform that while looking at the scene
from the top). During object shifting, holding down the shift key allows
to perform smaller shift steps. Holding down the ctrl key allows to move 
in an orthogonal direction to the regular direction(s). When done, select
the camera pan toolbar button again:

\figx{fig/bubble/10.jpg}{height=0.1\textheight}

We set a target velocity of 50 for the left motor and run the simulation: 
the graph view now displays the distance to the closest obstacle and the 
distance segment is visible in the scene too. We stop the simulation and
reset the target velocity to zero.

We now need to finish BubbleRob as a model definition. We select the model 
base (i.e. object bubbleRob) then check items Object is model base and 
Object/model can transfer or accept DNA in the object common properties: 
there is now a stippled bounding box that encompasses all objects in the
model hierarchy. We select the two joints, the proximity sensor and the 
graph, then enable item Don't show as inside model selection and click 
Apply to selection, in the same dialog: the model bounding box now ignores
the two joints and the proximity sensor. Still in the same dialog, we 
disable camera visibility layer 2, and enable camera visibility layer 10
for the two joints and the force sensor: this effectively hides the two 
joints and the force sensor, since layers 9-16 are disabled by default. At
any time we can modify the visibility layers for the whole scene. To finish
the model definition, we select the vision sensor, the two wheels, the
slider, and the graph, then enable item Select base of model instead: if we
now try to select an object in our model in the scene, the whole model will
be selected instead, which is a convenient way to handle and manipulate the
whole model as a single object. Additionally, this protects the model 
against inadvertant modification. Individual objects in the model can still
be selected in the scene by click-selecting them with control-shift, or
normally selecting them in the scene hierarchy. We finally collapse the
model tree in the scene hierarchy. This is what we have:

\fig{\\BubbleRob model definition}{fig/bubble/11.jpg}{height=0.5\textheight}

Next we will add a vision sensor, at the same position and orientation as 
BubbleRob's proximity sensor. We open the model hierarchy again, then 
click [Menu bar --> Add --> Vision sensor --> Perspective type], then
attach the vision sensor to the proximity sensor, and set the local 
position and orientation of the vision sensor to (0,0,0). We also make 
sure the vision sensor is not not visible, not part of the model bounding
box, and that if clicked, the model will be selected instead. In order to
customize the vision sensor, we open its properties dialog. We set the Far
clipping plane item to 1, and the Resolution x and Resolution y items to 
256 and 256. We then open the vision sensor filter dialog by clicking Show
filter dialog. We select the filter component Edge detection on work image
and click Add filter. We position the newly added filter in second position
(one position up, using the up button). We double-click the newly added
filter component and adjust its Threshold item to 0.2, then click OK. We 
add a floating view to the scene, and over the newly added floating view,
right-click [Popup menu --> View --> Associate view with selected vision 
sensor] (we make sure the vision sensor is selected during that process). 
To be able to see the vision sensor's image, we start the simulation, 
then stop it again.

Next, we want to be able to modulate BubbleRob's velocity with a custom 
user interface. To open the custom user interface dialog we click the
appropriate toolbar button:

\figx{fig/bubble/12.jpg}{height=0.1\textheight}

We click Add new user interface and enter 2 for the Client y-size item 
then click OK. A new custom user interface was added to the scene (in the
middle of the page). We rename it to bubbleRob\_ui:

\fig{Custom user interface editor}{fig/bubble/13.jpg}{height=0.5\textheight}

Other custom user interfaces might be visible. We now shift-select all 
free cells of the custom user interface, and click Insert merged button:
a large button was placed over the selected cells. Notice the Button handle
item when the new button is selected: 3. This number can be used at a later
stage to programmatically access that button. If we wanted, we could change
that handle. We then select Slider as type. The button changed to a slider.
We deselect all cells, then select bubbleRob for item UI is associated with.
This is important so that the custom user interface is also automatically
copied (and saved) if bubbleRob is copied (or saved as a model). Finally, 
we enable item UI is visible only during simulation. We leave the custom 
user interface edit mode by closing its dialog. 

The last thing that we need for our scene is a small child script that will
control BubbleRob's behavior. We select bubbleRob and click [Menu bar --> 
Add --> Associated child script --> Non threaded]. This just added a 
non-threaded child script to the scene, and associated it with bubbleRob. 
We can also add, remove or modify scripts via the script dialog which can
be opened with [Menu bar --> Tools --> Scripts] or through the appropriate
toolbar button:

\figx{fig/bubble/14.jpg}{height=0.1\textheight}

We double-click the little script icon that appeared next to bubbleRob's 
name in the scene hierarchy: this opens the child script that we just added. 
We copy and paste following code into the script editor, then close it:

\lstx{bubble.lua}{lst/bubble.lua}

We run the simulation. BubbleRob now moves forward while trying to avoid 
obstacles (in a very basic fashion). While the simulation is still running,
change BubbleRob's velocity, and copy/paste it a few times. Also try to 
scale a few of them while the simulation is still running. Be aware that 
the minimum distance calculation functionality might be heavily slowing
down the simulation, depending on the environment. You can turn that 
functionality on and off in the distance dialog, by checking / unchecking
the Enable all distance calculations item.

Using a script to control a robot or model is only one way of doing. 
V-REP offers many different ways (also combined), have a look at the 
external controller tutorial.

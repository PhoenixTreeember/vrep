\secrel{Robot language interpreter integration}\label{langtutor}

This tutorial will try to explain how to integrate or embed a robot language 
interpreter into V-REP. The procedure is very similar in case you want to 
integrate an emulator (e.g. a specific microcontroller emulator) into V-REP. 
Extending V-REP's functionality requires most of the time the development 
of a plugin. Make sure you have read the tutorial on plugins\ \ref{ctlplugin}, 
and the tutorial on external controllers\ \ref{excontroller}\ before 
proceeding with this tutorial.

The V-REP scene file related to this tutorial is located in V-REP's 
installation folder \file{scenes/robotLanguageControl.ttt}. The plugin 
project files, and the server application project files of this tutorial 
are located in V-REP's installation folder \file{programming/v\_repExtMtb}\
and \file{programming/mtbServer}. The MTB plugin and MTB server should 
already be compiled in the installation directory.

\clearpage
First, let's start by loading the related scene file 
\file{scenes/robotLanguageControl.ttt}:

\figx{fig/langtut/01.jpg}{height=0.95\textheight}

The MTB robot is an imaginary robot (MTB stands for Machine Type B), that 
will be controlled with an \term{imaginary robot language}.

As previously stated, the used robot language is imaginary and very very 
simple. Following commands are supported (one command per line, input is 
case-sensitive):

\lstx{langtut.imag}{lst/langtut.imag}

Any word different from "REM", "SETLINVEL", "SETROTVEL", "MOVE", "WAIT", 
"SETBIT", "CLEARBIT", "IFBITGOTO", "IFNBITGOTO" and "GOTO" is considered to 
be a label. Now run the simulation. If the related plugin was not found, 
following message displays (the display of the message is handled in the 
child scripts attached to objects MTB\_Robot and MTB\_Robot\#0):

\figx{fig/langtut/02.jpg}{height=0.3\textheight}

If the related plugin was found, then the the MTB plugin will launch a 
server application (i.e. mtbServer) that basically represents the robot 
language interpreter and controller. There is no direct need for a server 
application, the mtbServer functionality could also be directly running 
inside of the MTB plugin. The main advantages of using that functionality 
inside of a server application are:
\begin{itemize}[nosep]
	\item The MTB plugin can act as intermediate for as many different 
		languages as needed, also those that haven't been developed yet: 
		the MTB plugin will simply launch the appropriate server depending 
		on the used robot/language.
	\item If the robot language interpreter / controller crashes, \vrep\
		won't crash, since the two are distinct and separate processes.
\end{itemize}

Currently, the MTB server is in charge of two main tasks:
\begin{itemize}[nosep]
	\item receive the program code (i.e. a buffer) from the MTB plugin, 
		compile it, and initialize the robot controller.
	\item apply input signals, step through the program code (the step 
		duration can be different from step to step), and return output 
		signals and joint angles.
\end{itemize}

If the MTB server detects an error during compilation of the program code, 
it will return an error message to the plugin, that will hand it over to the 
calling child script (i.e. in our case, the child scripts attached to 
objects MTB\_Robot and MTB\_Robot\#0.), which will display (for example):

\figx{fig/langtut/03.jpg}{height=0.25\textheight}

If the compilation was successful, then the robots start executing their 
respective program. The simulation is a maximum speed simulation, but can 
be switched to real-time simulation by toggling the related toolbar button:

\figx{fig/langtut/04.jpg}{height=0.1\textheight}

The execution speed can be even more accelerated by pressing the appropriate
toolbar button several times:

\figx{fig/langtut/05.jpg}{height=0.1\textheight}

Each MTB robot program can be individually paused, stopped or restarted at 
any time via their displayed custom dialog, which are custom user interfaces:

\figx{fig/langtut/06.jpg}{height=0.3\textheight}

Above custom user interface is the user-interface of the MTB robot and can 
be fully customized. Should the MTB robot be copied, then its custom user 
interface will also be copied. Next to being able to controlling the program
execution state, the custom user interface also displays current program 
line (Cmd) and the MTB's current joint values. The button located at the 
bottom of the custom user interface\note{show input/output}\ allows toggling
the input/output dialog:

\figx{fig/langtut/07.jpg}{height=0.3\textheight}

The above custom user interface allows the user to change the robot's input 
port bits, and to read the robot's output port bits. Input and output ports
can be read and respectively written by the robot language program. Input 
and output ports can also be written and read by external devices\note{e.g. 
the robot's gripper or suction pad}\ by using appropriate function calls
\note{see further below}.

There are two child scripts attached to the MTB\_Robot and MTB\_Robot\#0 
objects. They are in charge of handling the custom dialogs\note{custom user 
interfaces}\ and communicating with the MTB plugin. Most code in the child 
scripts could be handled by the plugin too. Open the child script attached 
to one of the two MTB robot\note{e.g. with a double-click on the script icon 
next to the robot model in the scene hierarchy}. At the top of the script,
you will see the robot language code.

Try to modify an MTB robot's program to have it perform a different 
movement sequence. Experiment a little bit.

The MTB robots are handled in following way:
\begin{itemize}[nosep]
	\item the actual robot language program is compiled and executed by 
		the "mtbServer"\ application. That application also holds the MTB 
		robot's state variables. For each MTB robot in the simulation scene, 
		there will be an instance of the mtbServer application launched by 
		the v\_repExtMtb plugin.
	\item the v\_repExtMtb plugin is in charge of providing custom Lua 
		functions, and also launches the mtbServer application when needed, 
		and communicates with it via socket communication.
	\item the child scripts attached to MTB\_Robot and MTB\_Robot\#0 check 
		whether the v\_repExtMtb plugin is loaded, update the custom user
		interfaces for each robot, and handle the communication with the plugin.
\end{itemize}

The MTB robot and its simple robot language is a simple prototype meant to 
demonstrate how to integrate a robot language interpreter into V-REP. It is 
very easy to extend current functionality for much more complex robots or
robot languages. All what is needed is:
\begin{itemize}[nosep]
	\item Building the model of the robot. This includes importing CAD data, 
		adding joints, etc. This step can be entirely done in V-REP.
	\item Writing a plugin to handle the new robot natively, i.e. to handle 
		the new robot by interpreting its own robot language. Any language 
		capable of accessing C-API functions and capable of being wrapped 
		in a dll can be used to create the plugin (but c/c++ is preferred). 
		The robot language interpreter could be directly embedded in the 
		plugin, or launched as an external application (mtbServer) as is 
		done in this tutorial.
	\item Writing a small child script responsible for handling custom 
		dialogs and linking the robot with the plugin. This step can be 
		entirely done in V-REP.
\end{itemize}

Now let's have a look at the MTB's plugin project. There is one 
prerequisites to embedding a robot language interpreter (or other 
emulator) into V-REP:
\begin{itemize}[nosep]
	\item The robot language interpreter should be able to be executed 
		several times in parallel. This means that several interpreter 
		instances should be supported, in order to support several 
		identical, in-parallel operating robots. This can be handled the 
		easiest by launching a new interpreter for each new robot, as is 
		done in this tutorial.
\end{itemize}

When writing any plugin, make sure that the plugin accesses V-REP's regular 
API only from the main thread (or from a thread created by V-REP)! The plugin
can launch new threads, but in that case those new threads should not be 
used to access V-REP (they can however be used to communicate with a server 
application, to communicate with some hardware, to execute background 
calculations, etc.).

\bigskip

Now let's have a look at the child script that is attached to the MTB robot. 
The code might seem quite long or complicated. However most functionality 
handled in the child script could also be directly handled in the plugin, 
making the child script much smaller/cleaner. The advantage in handling 
most functionality in the child script is that modifications can be 
performed without having to recompile the plugin!

Following is the MTB robot's child script main functionality:
\begin{itemize}[nosep]
	\item Checking whether the plugin was loaded. If not, an error 
		message is output.
	\item Communicating with the plugin. This means that information is 
		sent to and received from the MTB plugin with custom Lua functions.
	\item Applying the newly calculated joint values to the MTB robot model. 
		This could also be handled in the MTB's plugin.
	\item Reacting to events on the custom dialogs (custom user 
		interfaces), like button presses. This could also be handled in 
		the MTB's plugin.
	\item Updating the state of the custom dialogs (custom user interfaces), 
		by changing their visual appearance (e.g. displaying the current
		joint positions, etc.). This could also be handled in the MTB's plugin.
\end{itemize}

Following 3 custom Lua functions are of main interest (others are exported 
by the plugin): 

\begin{lstlisting}
number mtbServerHandle,string message=simExtMtb_startServer(string mtbServerExecutable,
    number portNumber,charBuffer program,table_4 jointPositions, table_2 velocities)

number result,string message=simExtMtb_step(number mtbServerHandle,number timeStep)

table_4 jointValues=simExtMtb_getJoints(number mtbServerHandle)
\end{lstlisting}
\begin{description}[nosep]
	\item{\textbf{simExtMtb\_startServer}}
		launches the server application (e.g. mtbServer) on the specified 
		port, connects to it, and sends it the robot language code, the 
		initial joint positions, and the initial velocities. In return,
		the function returns a server handle (if successful), and a message
		(usually a compilation error message).
	\item{\textbf{simExtMtb\_step}}
		steps through the robot language program with the specified timeStep,
		and returns a result value and a message (usually the code being
		currently executed).
	\item{\textbf{simExtMtb\_getJoints}}
		retrieves the current joint positions. The joint positions are 
		automatically updated when simExtMtb\_step is called.
\end{description}

You could also imagine slightly modifying the step function, and add one 
additional function, in order to be able to handle intermediate events 
triggered by the robot language program execution. In that case, each 
simulation step would have to execute following script code (in a 
non-threaded child script):

\begin{lstlisting}
local dt=simGetSimulationTimeStep()
while (dt>0) do
    result,dt,cmdMessage=simExtMtb_step(mtbServerHandle,dt) 
		-- where the returned dt is the remaining dt
    local event=simExtMtb_getEvent()
    while event~=-1 do
        -- handle events here
        event=simExtMtb_getEvent()
    end
end
\end{lstlisting}

